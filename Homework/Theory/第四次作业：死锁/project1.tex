\documentclass{article}
\usepackage{fancyhdr}
\usepackage{ctex}
\usepackage{listings}
\usepackage{graphicx}
\usepackage[a4paper, body={18cm,22cm}]{geometry}
\usepackage{amsmath,amssymb,amstext,wasysym,enumerate,graphicx}
\usepackage{float,abstract,booktabs,indentfirst,amsmath}
\usepackage{array}
\usepackage{booktabs}
\usepackage{multirow}
\usepackage{url}
\usepackage{diagbox}
\renewcommand\arraystretch{1.4}
\usepackage{indentfirst}
\setlength{\parindent}{2em}
\usepackage{enumitem}
\setmonofont{Consolas}
\usepackage{listings}
\usepackage{xcolor}
\usepackage{makecell}
\usepackage{tikz}
\usetikzlibrary{positioning, arrows.meta}
\setCJKmonofont{黑体}
\lstset{  
	% 基本设置  
	xleftmargin = 3em, xrightmargin = 3em, aboveskip = 1em,  
	backgroundcolor = \color{white},  
	basicstyle = \small\ttfamily,  
	rulesepcolor = \color{gray},  
	breaklines = true,  
	numbers = left,  
	numberstyle = \small,  
	numbersep = -14pt,  
	frame = shadowbox,  
	showspaces = false,  
	columns = fixed,  
	sensitive = true,  
	% VSCode 风格配色  
	keywordstyle = \color{blue!70!black}\bfseries,  
	emphstyle = \color{red!70!black}\bfseries, % 对于强调的词  
	emphstyle=[2]\color{purple!70!black}\bfseries, % 对于第二组强调的词  
	commentstyle = \color{green!60!black}, % 注释颜色  
	stringstyle = \color{orange!90!black}, % 字符串颜色更亮一些  
	morekeywords={ASSERT, int64\_t, uint32\_t},  
	moreemph={ASSERT, NULL},  
	moreemph=[2]{int64\_t, uint32\_t, tid\_t, uint8\_t, int16\_t, uint16\_t, int32\_t, size\_t, bool},  
	morecomment=[l][\color{green!60!black}]{+}, % 以+开头的注释  
}

%--------------------页眉--------------------%
\pagestyle{fancy}
\fancyhead[L]{}
\fancyhead[R]{}
\fancyhead[C]{华东师范大学软件工程学院}
\fancyfoot[C]{-\thepage-}
\renewcommand{\headrulewidth}{1.5pt}
%--------------------标题--------------------%
\begin{document}
\begin{center}
	{\Large{\textbf{\heiti 第四次作业:死锁}}}
	\begin{table}[H]
		\centering
		\begin{tabular}{p{2cm}p{4cm}<{\centering}p{1cm}p{2cm}p{6cm}<{\centering}}
			课程名称:    & 操作系统 & \quad & 指导教师:    & 张民
			\\ \cline{2-2} \cline{5-5}
			姓\qquad 名: & 王海生    & \quad & 学\qquad 号: & 10235101559         \\ \cline{2-2} \cline{5-5}
		\end{tabular}
	\end{table}
	
	% 添加新行并居中
	%\vspace{1em} % 可选:添加垂直间距
\end{center}
\rule{\textwidth}{1pt}

\tableofcontents

%--------------------正文--------------------%
\section{问题7.7}

\subsection{题目}

假设一个系统有 4 个相同类型的资源,并由 3 个进程共享。每个进程最多需要 2 个资源。证明这个系统不会死锁。

\subsection{解答}

假设系统有 4 个相同类型的资源,且由 3 个进程共享。每个进程最多需要 2 个资源。我们可以设这 3 个进程为 P1、P2 和 P3,每个进程的最大资源需求为 2 个,系统总共有 4 个资源。在任意情况下,无论进程的执行顺序如何,都能够保证每个进程请求的资源数(2 个)始终小于或等于当前可用资源与之前进程已占用资源的和。因为每个进程最多占用 2 个资源,总资源数为 4,这意味着在任何时刻,总会有足够的资源保证至少一个进程能够完成其操作,并释放资源供其他进程使用。因此,系统始终处于安全状态,必然不会发生死锁。

\section{问题7.12}

\subsection{题目}

假设一个系统具有如下的快照:

\begin{center}
	\begin{tabular}{|c|c|c|c|c|c|c|c|c|}
		\hline
		& \multicolumn{4}{|c|}{Allocation} & \multicolumn{4}{|c|}{Max} \\
		\cline{2-9}
		& A & B & C & D & A & B & C & D \\
		\hline
		$P_0$ & 3 & 0 & 1 & 4 & 5 & 1 & 1 & 7 \\
		\hline
		$P_1$ & 2 & 2 & 1 & 0 & 3 & 2 & 1 & 1 \\
		\hline
		$P_2$ & 3 & 1 & 2 & 1 & 3 & 3 & 2 & 1 \\
		\hline
		$P_3$ & 0 & 5 & 1 & 0 & 4 & 6 & 1 & 2 \\
		\hline
		$P_4$ & 4 & 2 & 1 & 2 & 6 & 3 & 2 & 5 \\
		\hline
	\end{tabular}
\end{center}

采用银行家算法,确定如下每个状态是否安全的。如果状态是安全的,那么说明进程可以完成的顺序。否则,说明为什么状态是不安全的。

a. Available = (0, 3, 0, 1)

b. Available = (1, 0, 0, 2)

\subsection{解答}

根据所学的知识:

\[
\text{need} = \text{max} - \text{allocation}
\]

\[
\text{need} = 
\begin{pmatrix}
	2 & 2 & 1 & 1 \\
	2 & 1 & 3 & 1 \\
	0 & 2 & 1 & 3 \\
	0 & 1 & 1 & 2
\end{pmatrix}
\]

\subsubsection{a小问}

\begin{enumerate}
	\item 
	计算需求矩阵(need = max - allocation):
	
	$$
	\begin{array}{|c|c|c|c|c|}
		\hline
		& A & B & C & D \\
		\hline
		P_0 & 2 & 1 & 0 & 3 \\
		\hline
		P_1 & 1 & 0 & 0 & 1 \\
		\hline
		P_2 & 0 & 2 & 0 & 0 \\
		\hline
		P_3 & 4 & 1 & 0 & 2 \\
		\hline
		P_4 & 2 & 1 & 1 & 3 \\
		\hline
	\end{array}
	$$
	
	\item
	初始化 \texttt{work = available = (0, 3, 0, 1)},\texttt{finish = (0, 0, 0, 0, 0)},表示所有进程都未完成。
	
	\item 
	依次检查进程是否能完成:
	
	\begin{enumerate}
		\item 第一步:检查进程 $P_2$,其需求为 $(0, 2, 0, 0)$,满足 \texttt{need[i] <= work},即 $(0, 2, 0, 0) \leq (0, 3, 0, 1)$,因此可以完成 $P_2$。
		\begin{itemize}
			\item 更新 \texttt{work = work + allocation[2] = (0, 3, 0, 1) + (3, 1, 2, 1) = (3, 4, 2, 2)},更新 \texttt{finish = (0, 0, 1, 0, 0)}。
		\end{itemize}
		
		\item 第二步:检查进程 $P_1$,其需求为 $(1, 0, 0, 1)$,满足 \texttt{need[i] <= work},即 $(1, 0, 0, 1) \leq (3, 4, 2, 2)$,因此可以完成 $P_1$。
		\begin{itemize}
			\item 更新 \texttt{work = work + allocation[1] = (3, 4, 2, 2) + (2, 2, 1, 0) = (5, 6, 3, 2)},更新 \texttt{finish = (0, 1, 1, 0, 0)}。
		\end{itemize}
		
		\item 第三步:检查进程 $P_3$,其需求为 $(4, 1, 0, 2)$,满足 \texttt{need[i] <= work},即 $(4, 1, 0, 2) \leq (5, 6, 3, 2)$,因此可以完成 $P_3$。
		\begin{itemize}
			\item 更新 \texttt{work = work + allocation[3] = (5, 6, 3, 2) + (0, 5, 1, 0) = (5, 11, 4, 2)},更新 \texttt{finish = (0, 1, 1, 1, 0)}。
		\end{itemize}
		
		\item 第四步:检查进程 $P_0$,其需求为 $(2, 1, 0, 3)$,满足 \texttt{need[i] <= work},即 $(2, 1, 0, 3) \leq (5, 11, 4, 2)$,因此可以完成 $P_0$。
		\begin{itemize}
			\item 更新 \texttt{work = work + allocation[0] = (5, 11, 4, 2) + (3, 0, 1, 4) = (8, 11, 5, 6)},更新 \texttt{finish = (1, 1, 1, 1, 0)}。
		\end{itemize}
		
		\item 第五步:检查进程 $P_4$,其需求为 $(2, 1, 1, 3)$,不满足 \texttt{need[i] <= work},即 $(2, 1, 1, 3) \nleq (8, 11, 5, 6)$,因此无法完成 $P_4$。
	\end{enumerate}
	
	\item 
	由于在迭代结束时,\texttt{finish} 不全为 true,系统处于非安全状态。对 $P_4$ 的请求可能导致死锁,因此该系统是非安全的。
\end{enumerate}

\subsubsection{b小问}

\begin{enumerate}
	\item 计算需求矩阵(need = max - allocation):
	
	$$
	\begin{array}{|c|c|c|c|c|}
		\hline
		& A & B & C & D \\
		\hline
		P_0 & 2 & 1 & 0 & 3 \\
		\hline
		P_1 & 1 & 0 & 0 & 1 \\
		\hline
		P_2 & 0 & 2 & 0 & 0 \\
		\hline
		P_3 & 4 & 1 & 0 & 2 \\
		\hline
		P_4 & 2 & 1 & 1 & 3 \\
		\hline
	\end{array}
	$$
	
	\item 
	初始化 \texttt{work = available = (1, 0, 0, 2)},\texttt{finish = (0, 0, 0, 0, 0)},表示所有进程都未完成。
	
	\item 
	依次检查进程是否能完成:
	
	\begin{enumerate}
		\item 第一步:检查进程 $P_1$,其需求为 $(1, 0, 0, 1)$,满足 \texttt{need[i] <= work},即 $(1, 0, 0, 1) \leq (1, 0, 0, 2)$,因此可以完成 $P_1$。
		\begin{itemize}
			\item 更新 \texttt{work = work + allocation[1] = (1, 0, 0, 2) + (2, 2, 1, 0) = (3, 2, 1, 2)},更新 \texttt{finish = (0, 1, 0, 0, 0)}。
		\end{itemize}
		
		\item 第二步:检查进程 $P_0$,其需求为 $(2, 1, 0, 3)$,满足 \texttt{need[i] <= work},即 $(2, 1, 0, 3) \leq (3, 2, 1, 2)$,因此可以完成 $P_0$。
		\begin{itemize}
			\item 更新 \texttt{work = work + allocation[0] = (3, 2, 1, 2) + (3, 0, 1, 4) = (6, 2, 2, 6)},更新 \texttt{finish = (1, 1, 0, 0, 0)}。
		\end{itemize}
		
		\item 第三步:检查进程 $P_2$,其需求为 $(0, 2, 0, 0)$,满足 \texttt{need[i] <= work},即 $(0, 2, 0, 0) \leq (6, 2, 2, 6)$,因此可以完成 $P_2$。
		\begin{itemize}
			\item 更新 \texttt{work = work + allocation[2] = (6, 2, 2, 6) + (3, 1, 2, 1) = (9, 3, 4, 7)},更新 \texttt{finish = (1, 1, 1, 0, 0)}。
		\end{itemize}
		
		\item 第四步:检查进程 $P_3$,其需求为 $(4, 1, 0, 2)$,满足 \texttt{need[i] <= work},即 $(4, 1, 0, 2) \leq (9, 3, 4, 7)$,因此可以完成 $P_3$。
		\begin{itemize}
			\item 更新 \texttt{work = work + allocation[3] = (9, 3, 4, 7) + (0, 5, 1, 0) = (9, 8, 5, 7)},更新 \texttt{finish = (1, 1, 1, 1, 0)}。
		\end{itemize}
		
		\item 第五步:检查进程 $P_4$,其需求为 $(2, 1, 1, 3)$,满足 \texttt{need[i] <= work},即 $(2, 1, 1, 3) \leq (9, 8, 5, 7)$,因此可以完成 $P_4$。
		\begin{itemize}
			\item 更新 \texttt{work = work + allocation[4] = (9, 8, 5, 7) + (4, 2, 1, 2) = (13, 10, 6, 9)},更新 \texttt{finish = (1, 1, 1, 1, 1)}。
		\end{itemize}
		
		\item 
		所有进程都能够完成,\texttt{finish} 全部为 true,因此该系统是安全的。
	\end{enumerate}
	
	\textbf{安全序列:} $P_1, P_0, P_2, P_3, P_4$
\end{enumerate}

\section{问题7.13}

\subsection{题目}

假设一个系统具有如下的快照:

\begin{center}
	\begin{tabular}{|c|c|c|c|c|c|c|c|c|c|c|c|c|}
		\hline
		Process & \multicolumn{4}{|c|}{Allocation} & \multicolumn{4}{|c|}{Max} & \multicolumn{4}{|c|}{Available} \\
		\cline{2-13}
		& A & B & C & D & A & B & C & D & A & B & C & D \\
		\hline
		$P_0$      & 2 & 0 & 0 & 1 & 4 & 2 & 1 & 2 & & & & \\
		\hline
		$P_1$      & 3 & 1 & 2 & 1 & 5 & 2 & 5 & 2 & & & & \\
		\hline
		$P_2$      & 2 & 1 & 0 & 3 & 2 & 3 & 1 & 6 & & & & \\
		\hline
		$P_3$      & 1 & 3 & 1 & 2 & 1 & 4 & 2 & 4 & & & & \\
		\hline
		$P_4$      & 1 & 4 & 3 & 2 & 3 & 6 & 6 & 5 & & & & \\
		\hline
	\end{tabular}
\end{center}

采用银行家算法,回答下面的问题:
a. 通过进程可以完成执行的顺序,说明系统处于安全状态。
b. 当进程 $P_1$ 的请求为 $(1, 1, 0, 0)$ 时,能否立即允许这一请求?
c. 当进程 $P_4$ 的请求为 $(0, 0, 2, 0)$ 时,能否立即允许这一请求?

\subsection{解答}

\subsubsection{a小问}

\begin{enumerate}
	\item 初始化 \texttt{work = available = (3, 3, 2, 1)},\texttt{finish = (0, 0, 0, 0, 0)},表示所有进程都未完成。
	
	\item 依次检查进程是否能完成:
	
	\begin{enumerate}
		\item 第一步:检查进程 $P_0$,其需求为 \texttt{need[0]},满足 \texttt{need[0] <= work},因此可以完成 $P_0$。
		\begin{itemize}
			\item 更新 \texttt{work = work + allocation[0] = (3, 3, 2, 1) + (2, 0, 0, 1) = (5, 3, 2, 2)},更新 \texttt{finish = (1, 0, 0, 0, 0)}。
		\end{itemize}
		
		\item 第二步:检查进程 $P_3$,其需求为 \texttt{need[3]},满足 \texttt{need[3] <= work},因此可以完成 $P_3$。
		\begin{itemize}
			\item 更新 \texttt{work = work + allocation[3] = (5, 3, 2, 2) + (1, 3, 1, 2) = (6, 6, 3, 4)},更新 \texttt{finish = (1, 0, 0, 1, 0)}。
		\end{itemize}
		
		\item 第三步:检查进程 $P_1$,其需求为 \texttt{need[1]},满足 \texttt{need[1] <= work},因此可以完成 $P_1$。
		\begin{itemize}
			\item 更新 \texttt{work = work + allocation[1] = (6, 6, 3, 4) + (3, 1, 2, 1) = (9, 7, 5, 5)},更新 \texttt{finish = (1, 1, 0, 1, 0)}。
		\end{itemize}
		
		\item 第四步:检查进程 $P_2$,其需求为 \texttt{need[2]},满足 \texttt{need[2] <= work},因此可以完成 $P_2$。
		\begin{itemize}
			\item 更新 \texttt{work = work + allocation[2] = (9, 7, 5, 5) + (2, 1, 0, 3) = (11, 8, 5, 8)},更新 \texttt{finish = (1, 1, 1, 1, 0)}。
		\end{itemize}
		
		\item 第五步:检查进程 $P_4$,其需求为 \texttt{need[4]},满足 \texttt{need[4] <= work},因此可以完成 $P_4$。
		\begin{itemize}
			\item 更新 \texttt{work = work + allocation[4] = (11, 8, 5, 8) + (1, 4, 3, 2) = (12, 12, 8, 10)},更新 \texttt{finish = (1, 1, 1, 1, 1)}。
		\end{itemize}
		
		\item 
		迭代结束,\texttt{finish} 对所有进程均为 true,系统处于安全状态。一个可完成的进程执行顺序为 \textbf{<P0, P3, P1, P2, P4>}。
	\end{enumerate}
\end{enumerate}

\subsubsection{b小问}

\begin{enumerate}
	\item 检查进程 $P_1$ 的请求 \texttt{request[1] = (1, 1, 0, 0)},满足 \texttt{request[1] <= need[1]} 和 \texttt{request[1] <= available}。
	\begin{itemize}
		\item 更新 \texttt{available = available - request[1] = (2, 2, 2, 1)}。
		\item 更新 \texttt{allocation[1] = allocation[1] + request[1] = (4, 2, 2, 1)}。
		\item 更新 \texttt{need[1] = need[1] - request[1] = (1, 0, 3, 1)}。
	\end{itemize}
	\item 修改状态后,经检验系统仍为安全状态,则请求可被立刻允许。
\end{enumerate}

\subsubsection{c小问}

\begin{enumerate}
	\item 检查进程 $P_4$ 的请求 \texttt{request[4] = (0, 0, 2, 0)},满足 \texttt{request[4] <= need[4]} 和 \texttt{request[4] <= available}。
	\begin{itemize}
		\item 更新 \texttt{available = available - request[4] = (3, 3, 0, 1)}。
		\item 更新 \texttt{allocation[4] = allocation[4] + request[4] = (1, 4, 5, 2)}。
		\item 更新 \texttt{need[4] = need[4] - request[4] = (2, 2, 1, 3)}。
	\end{itemize}
	\item 修改状态后,经检验系统为非安全状态(安全算法第一步就无法找到 i 使得 \texttt{need[i] <= work = available}),则该请求不能立刻被允许,进程 $P_4$ 应该等待该请求且系统恢复到原来的资源分配状态。
\end{enumerate}

\end{document}